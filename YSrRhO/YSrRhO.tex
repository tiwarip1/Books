\documentclass[garamond]{article}
\usepackage[version=4]{mhchem}
\usepackage{mathrsfs,relsize,makeidx,color,setspace,amsmath,amsfonts,amssymb}
\usepackage[table]{xcolor}
\usepackage{bm,ltablex,microtype}
\usepackage{placeins}
\usepackage{listings}
\usepackage[utf8]{inputenc}
\usepackage[top = 1in, bottom = 1in, right = 1in, left = 1in]{geometry}
\usepackage[pdftex]{graphicx}
\usepackage{blindtext}
\usepackage{MnSymbol,wasysym}
\usepackage{calrsfs}
\usepackage[mathscr]{euscript}
\usepackage{hyperref}

\newlength\tindent
\setlength{\tindent}{\parindent}
\setlength{\parindent}{0pt}
\renewcommand{\indent}{\hspace*{\tindent}}

\title{YSrRhO$_4$}
\author{Pranjal Tiwari}

\begin{document}

\maketitle

A crystal of $La_{.5}Sr_{1.5}RhO_4$ was produced using the Flux Free Floating Zone Method and was shown to have semi-conducting properties regarding resistivity as it changes with temperature.$^{\cite{LaSrRhO}}$ If one were to look at the physical structure of this material, it would appear similar to $YBaCuO$, which is a superconductor at around liquid nitrogen temperatures. The crystal field of the material in question looks like Figure \ref{fig:1}.

\begin{figure}[ht]
	\begin{center}
		\includegraphics[width=.5\columnwidth]{figures/crystal_field.jpg}
		\caption{Crystal field of the d-orbital of Rh with occupation by 7 electrons}
		\label{fig:1}
	\end{center}
\end{figure}

Since Rhodium has 7 valence electrons in its outer d-shell, it can occupy up to the $d_{z^2}$ orbital which would have it interact with orbitals along the c-axis, which is not conducive to conductors. However, if pressure were to be applied uniaxially along the c-axis, the energy of the $d_{z^2}$ orbital would increase till is surpasses the $d_{x^2-y^2}$ orbital. Once this occurs, electrons will be transferred to the lower energy state, which is along the ab-plane, which is a common property of conductors.\\

This would appear to be a Mott Insulator transition, but the experiment would have to be done to give any credibility to this statement. If this were possible, uniaxial pressure can be applied through the use of an anvil or through chemical pressure by replacing atoms with smaller replacements. The $La$ in this case would be replaced with a $Y$ atom, as it they have the same electron structure; however, the production of this material has not yet been published, so we have no knowledge as to whether it can even be produced. 

\begin{thebibliography}{9}
	\bibitem{LaSrRhO}
	https://journals.aps.org/prmaterials/abstract/10.1103/PhysRevMaterials.1.044005
\end{thebibliography}

\end{document}